\subsection{Visión global del sistema}
Como ya se ha comentado en los capítulos anteriores, el proyecto consiste en el desarrollo de API rest-ful desde la que poder gestionar un repositorio de árboles genealógicos.

Para ello el \textit{backend} implementado deberá permitir que diferentes aplicaciones se den de alta para poder acceder a los datos almacenados por los usuarios y así mismo permitir a estos usuarios 
realizar las operaciones necesarias para gestionar sus árboles genealógicos.

\subsection{Actores}
Los actores implicados en el uso del sistema son dos:

\begin{description}
\item[Usuario final]:\newline
Este actor sera el que se dará de alta en nuestra aplicación y ha de ser capaz de almacenar y gestionar sus árboles genealógicos.

 
\item[Aplicación]:\newline		
El \textit{backend} ha de permitir a diferentes aplicaciones (\textit{fontends}) darse de alta. Estas aplicaciones recibirán las credenciales necesarias para poder conseguir los permisos de acceso de nuestros usuarios finales y manipular sus árboles.
\end{description}

\subsection{Requisitos funcionales}
Los requisitos funcionales describen todas las funcionalidades que proporcionara el \textit{backend}. A continuación se listan todos los requisitos funcionales desglosados por actor:

\subsubsection{Usuario final}
\begin{description}
\item[Gestión de cuenta]:
	\begin{description}
	\item[Registro]:
	El usuario deberá poderse registrar en el sistema introduciendo los datos necesarios.
	\item[Modificar sus datos]:
	El usuario deberá poder modificar sus datos.
	\item[Login]:
	El usuario deberá de poder acceder al una interfaz desde la que gestionar sus datos de usuario.
	\item[Activación de cuenta]:
	El usuario una vez registrado tendrá que poder activar la cuenta vía email.
\end{description}
\end{description}		


\subsubsection{Aplicación}
\begin{description}
\item[Gestión de cuenta]:
	\begin{description}
	\item[Registro]:
	El usuario deberá poderse registrar en la aplicación introduciendo los datos necesarios.
	\item[Login]:
	El usuario deberá de poder acceder al una interfaz desde la que gestionar sus datos de usuario.
	\item[Modificar sus datos]:
	El usuario deberá poder modificar sus datos.
	\end{description}		
\end{description}
\begin{description}
\item[Gestión de aplicaciones]:
	\begin{description}
	\item[Dar de alta aplicación]:
	El usuario ha de poder dar de alta aplicaciones que posteriormente accederán al contenido del repositorio mediante la API.
	\item[Conseguir token]:
	Las aplicaciones una vez dadas de alta podrán pedir \textit{tokens} asignados a los usuarios finales con su nivel autorización y que permita su autentificación.
	\end{description}
	
\begin{description}
\item[Gestión de los árboles]:
	\begin{description}
	\item[Creación, modificación y consulta de árboles genealógicos]:
	El usuario deberá poder crear, modificar y consultar los árboles genealógicos.
	\item[Creación, modificación y consulta de personas]:
	El usuario deberá de poder crear, modificar y consultar las personas almacenadas en los árboles genealógicos.
	\item[Creación, modificación y consulta de eventos]:
	El usuario deberá de poder crear, modificar y consultar los eventos de las personas almacenadas en los árboles genealógicos.
	\item[Subir un archivo GEDCOM]:
	El usuario deberá de poder cargar archivos GEDCOM en su cuenta.
	\item[Encontrar personas similares]:
	El usuario deberá de poder seleccionar una persona de uno de sus árboles genealógicos y buscar personas similares en otros árboles de otros usuarios.
	\end{description}
\end{description}
\end{description}

\subsection {Requisitos no funcionales}
A continuación se muestra una lista de los requisitos no funcionales del sistema:

\begin{enumerate}
\item La API ha de permitir un uso completo de todas las funcionalidades independientemente de la tecnología que la use.
\item Se ha de proporcionar una interfaz gráfica para gestionar los usuarios de la aplicación.I.
\item El tiempo de respuesta de todos los endpoints ha de ser similar sea cuan sea la funcionalidad que implementen.
\item La plataforma ha de constar de sistemas que aseguren la seguridad de los datos de usuario.
\item Todos los \textit{endpoints} de la API tendrán que respetar los estandartes de una aplicación rest-ful
\end{enumerate}
