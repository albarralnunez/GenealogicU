\subsection{Contextualización del proyecto}

\subsubsection{Los arboles genealógicos}
Un arbole genealógico, también llamado genorama, es la representación gráfica de los antepasados  y descendientes de un individuo.Para su representación  se suelen usar tablas o arboles, siendo esta ultima la forma más común y la que se usara en el proyecto.

\subsubsection{Uso y aplicación de los arboles genealógicos.}
Los arboles genealógicos se usan como herramienta en la genealogía, que se encarga de estudiar y seguir la ascendencia y descendencia de una persona o familia. La genealogía es una ciencia auxiliar de la Historia y es trabajada por un genealogista. Uno de los objetivos del software a desarrollar es dar soporte a los genealogistas. \linebreak Por otro lado hay varias comunidades de aficionados que llevan sus propios arboles genealógicos, el software creado también les podrá dar servicio a esta tipología de usuarios.

\newpage
\subsection{Perspectiva general del software actual.}
Todo software genealógico, como mínimo permite almacenar la siguiente información de un individuo: fecha y lugar de nacimiento, fecha de casamiento, muerte y relaciones familiares, contra más flexible es el programa más información te permite introducir acerca de un individuo. También proporcionan diferentes maneras de representar la información y permiten exportar a GEDCOM la información representada.

\noindent\fbox{\parbox{\textwidth}{\begin{flushleft} GEDCOM \cite{aboutGEDCOM} (\textbf{G}enealogical \textbf{D}ata \textbf{COM}munication):\end{flushleft} Es un formato de archivo de datos, proporciona un formato flexible y uniforme para el intercambio de datos genealógicos computarizados.}}

La mayor parte del software genealógico actual esta basado en soluciones de escritorio, pero en los últimos años han proliferado diferentes soluciones web como myheritage o familysearch, que no solo sirven como plataforma de edición sino que también son grandes plataformas \textit{cloud} en las que se almacenan los arboles genealógicos.

Las soluciones más avanzadas, aparte de la gestión de arboles también ofrecen herramientas más orientadas a la investigación, como podrían ser sistemas de búsqueda de individuos basados en sus relaciones o herramientas estadísticas.